\documentclass{article}

%%% This is the template for CM10311, Discrete Mathematics and Databases, Problem Sheet 1.
%%% Please fill in your answers in the places provided, you do not need to make any other changes to the file.

\usepackage{amsmath} % for mathematical symbols
\usepackage{amssymb}
\usepackage{amsthm}
\usepackage{enumerate} % for fancy numbering
\usepackage[shortlabels]{enumitem}

\title{CM12004: Problem Sheet}
\author{} % don't add an author
\date{} % don't add a date, omitting this line makes today's date appear below the title

\begin{document}

\maketitle

\begin{enumerate}
  \item
  % question 1
  \begin{enumerate}[(i)]
    \item % (i)
% an example truth table...
%\begin{tabular}{r r l}
%$X$ & $Y$ & $X \land Y$ \\
%T & T & T \\
%T & F & F \\
%F & T & F \\
%F & F & F \\
 %\end{tabular}
          \begin{tabular}{|l|l|l|l|l|}
            \hline
            $X$ & $Y$ & $P$ & $Q$ & $P\to Q$\\ \hline
            0&0&0&1&1\\
            0&1&1&1&1\\
            1&0&1&1&1\\
            1&1&1&0&0\\ \hline
          \end{tabular}
          \textbf{conclusion:} neither

    \item % (ii)
          \begin{tabular}{|l|l|l|l|l|}
            \hline
            $X$ & $Y$ & $P$ & $Q$ & $P\to Q$\\ \hline
            0&0&0&1&1\\
            0&1&1&0&0\\
            1&0&1&0&0\\
            1&1&1&0&0\\ \hline
          \end{tabular}
          \textbf{conclusion:} neither
    \item % (iii)
          \begin{tabular}{|l|l|l|l|l|}
            \hline
            $X$ & $Y$ & $P$ & $Q$ & $P\to Q$\\ \hline
            0&0&1&1&1\\
            0&1&1&1&1\\
            1&0&0&0&1\\
            1&1&1&1&1\\\hline
          \end{tabular}
          \textbf{conclusion:} tautology
    \item % (iv)
          \begin{tabular}{|l|l|l|l|l|}
            \hline
            $X$ & $Y$ & $P$ & $Q$ & $P\to Q$\\ \hline
            0&0&1&1&1\\
            0&1&1&1&1\\
            1&0&1&1&1\\
            1&1&0&0&1\\\hline
          \end{tabular}
          \textbf{conclusion:} tautology
    \item % (v)
          \begin{tabular}{|l|l|l|l|l|l|}
            \hline
            $X$ & $Y$ & $Z$ & $P$ & $Q$ & $P\to Q$\\ \hline
            0&0&0&0&0&1\\
            0&0&1&0&0&1\\
            0&1&0&0&0&1\\
            0&1&1&0&1&1\\
            1&0&0&0&1&1\\
            1&0&1&1&1&1\\
            1&1&0&1&1&1\\
            1&1&1&1&1&1\\\hline
           \end{tabular}
          \textbf{conclusion:} tautology
    \item % (vi)
          \begin{tabular}{|l|l|l|l|l|}
            \hline
            $X$ & $Y$ & $P$ & $Q$ & $P\to Q$\\ \hline
            0&0&1&1&1\\
            0&1&1&1&1\\
            1&0&0&1&1\\
            1&1&1&0&0\\\hline
          \end{tabular}
          \textbf{conclusion:} neither
    \item % (vii)
          \begin{tabular}{|l|l|l|l|l|}
            \hline
            $X$ & $Y$ & $P$ & $Q$ & $P\to Q$\\ \hline
            0&0&1&0&0\\
            0&1&1&1&1\\
            1&0&0&0&1\\
            1&1&1&0&0\\\hline
          \end{tabular}
          \textbf{conclusion:} neither
    \item % (viii)
          \begin{tabular}{|l|l|l|l|l|l|}
            \hline
            $X$ & $Y$ & $Z$ & $P$ & $Q$ & $P\to Q$\\ \hline
            0&0&0&1&1&1\\
            0&0&1&1&1&1\\
            0&1&0&0&1&1\\
            0&1&0&0&1&1\\
            0&1&1&1&1&1\\
            1&0&0&0&0&1\\
            1&0&1&0&1&1\\
            1&1&1&0&0&1\\
            1&1&1&1&1&1\\\hline
          \end{tabular}
          \textbf{conclusion:} tautology
  \end{enumerate}

  \item
  % question 2
        \begin{enumerate}[(a)]
            \item
                binary logical connectives operate on 2 binary inputs. each binary input can have 2 values, 1 or 0.\\ \(\therefore\) there are 4 possible combinations of inputs that can be operated on by a connective.\\ \(\implies\) each logical connective can have 4 possible binary outputs, one corresponding to each possible input combination.\\ having 4 possible binary outputs \(\implies\) there are 16 possible combinations of outputs a connective can have.\\ \(\therefore\) \textbf{answer: there are 16 possible binary logical connectives.}
          \item
                \begin{enumerate}[(i)]
                  \item
                        \(\lnot(\lnot(P\land\lnot Q)\land\lnot(\lnot P\land Q))\)
                  \item
                        \(\lnot(\lnot P\lor Q)\lor\lnot(P\lor\lnot Q)\)
                  \item
                        \((P\to Q)\to\lnot(Q\to P)\)
                \end{enumerate}
                  \item
                        \def\lnand{\bar\land}
                        \begin{align}
                          \lnand&=\lnot(X\land Y)\\
                          \lnot&=X\lnand X\\
                          \land&=\lnot(\lnot(X\land Y))=(X\lnand Y)\lnand (X\lnand Y)\\
                          \lor&=\lnot(\lnot X\land\lnot Y)=(X\lnand X)\lnand(Y\lnand Y)\\
                          \to&=\lnot X\lor Y=\lnot(X\land\lnot Y)=X\lnand(Y\lnand Y)
          \end{align}
                  \item
                        \def\lnor{\bar\lor}
                        \begin{align}
                          \lnor&=\lnot(X\lor Y)\\
                          \lnot&=X\lnor X\\
                          \lor&=\lnot(\lnot(X\lor Y))=(X\lnor Y)\lnor (X\lnot Y)\\
                          \land&=\lnot(\lnot X\lor\lnot Y)=(X\lnor X)\lnor(Y\lnor Y)\\
                          \to&=\lnot X\lor Y=((X\lnor X)\lnor Y)\lnor((X\lnor X)\lnor Y)
          \end{align}

        \end{enumerate}

  \item
  % question 3
  \begin{enumerate}[(a)]
    \item % (a)
    \begin{enumerate}[(i)]
      \item % (i)
      \textbf{question: }\(\forall x\in\mathbb{Z}\exists y\in\mathbb{Z}(x^2<y+1)\)\\
      \textbf{true}
      \begin{proof}
      \begin{align}
       &x^2<y+1\\
       \implies& x^2-1<y\\
       \implies& y>x^2-1
      \end{align}
      if \(x\in \mathbb{Z}\) and \(y\in\mathbb{Z} \implies \exists y \mid y>x^2-1 \because \mathbb{Z}\) has no upper bound, so there will always be a bigger \(y\ \forall x\).
      \end{proof}
      \item % (ii)
      \textbf{question: }\(\exists x\in\mathbb{Z}\forall y\in\mathbb{Z}(x^2<y+1)\)\\
      \textbf{false}
      \begin{proof}
      \begin{align}
       &x^2<y+1\\
       \implies&x^2-1<y
      \end{align}
      we know this to be false, as for it to be true, $x^2-1$ would have to be smaller than every integer, which is impossible.
      \end{proof}
      \item % (iii)
      \textbf{question: }\(\exists x\in\mathbb{Z}\forall x\in\mathbb{Z}(x^2<y+1)\)\\\textbf{false}
      \begin{proof}
       \begin{align}
        &x^2<y+1\\
       \implies&x^2-1<y
       \end{align}
       for a similar reason to (ii) we know this to also be false, as for it to be true, $y$ would have to be bigger than every integer, which is impossible.
      \end{proof}
      \item
      \textbf{question: }\(\forall x\in\mathbb{Z}\exists y\in\mathbb{Z}((x<y)\to(x^2<y^2))\)\\\textbf{true}
      \begin{proof}
      let $y=x+1$.\\$x<y$ is true $\forall x,y\because x<x+1$\\
      \(\implies(x<y)\to(x^2<y^2)\) is true if $(x^2<y^2)$ is true.\\
      \(y^2=(x+1)^2=x^2+2x+1\implies y^2>x^2\)\\
      \(\therefore (x<y)\to(x^2<y^2)\) is true when $y=x+1$.\\
      \(\therefore \exists y\in\mathbb{Z} (y=x+1) \forall x\in\mathbb{Z}((x<y)\to(x^2<y^2))\)
      \end{proof}
    \end{enumerate}

    \item % (b)
    \begin{enumerate}[(i)]
      \item % (i)

      \item % (ii)

      \item % (iii)

    \end{enumerate}
  \end{enumerate}

  \item
  % question 4
  \begin{enumerate}[(i)]
    \item % (i)

    \item % (ii)

    \item % (iii)

    \item % (iv)

    \item % (v)

    \item % (vi)

    \item % (vii)

    \item % (viii)

  \end{enumerate}

  \item
  % question 5
  \begin{enumerate}[(a)]
    \item % (a)

    \item % (b)

    \item % (c)

  \end{enumerate}

  \item
  % question 6
  \begin{enumerate}[(a)]
    \item % (a)
    \begin{itemize}
      \item $2^{\emptyset}$

      \item $2^{\{0\}}$

      \item $2^{\{0\} \cup \{1\}}$

      \item $2^{\{\emptyset, 0, 1\}}$

      \item $2^{2^{2^{\{0,1\}}}}$

    \end{itemize}

    \item % (b)
    \begin{enumerate}[(i)]
      \item % (i)

      \item % (ii)

    \end{enumerate}

  \end{enumerate}
\end{enumerate}

\end{document}
